% This is abstract.tex
Tree volume measurement is one of the most common ways to categorize trees in forest invetory. For a long time measuring trees has been done manually by us humans. Information collected is put through specialized stem-profile taper functions that need to be made for each species in each defined area. This method of collecting and analyzing data is labour intensive, costly and leaves huge margin for error. This research project aims to solve that and focuses on developing the mathematical basis and computational algorithms for a new mobile application project started by SilviaTerra. The application makes use of phone sensor data paired with photogrammetric analysis and has been successful in accurately calculating the volume of a tree when the person is standing at a reasonable distance from the tree.